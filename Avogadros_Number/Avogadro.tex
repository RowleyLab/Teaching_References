\documentclass[12pt, openany, letterpaper]{memoir}
\usepackage{NotesStyle}
%\renewcommand\thesection{\thechapter\Alph{section}}
%\renewcommand\thesubsection{\thesection.\Numeral{subsection}}
\usepackage{xcolor}

\begin{document}
\begin{center}
{\Large Avogadro's Number and the Atomic Mass Unit}

Prepared by Dr. Matthew Rowley, 2019
\end{center}

\noindent
Our last quiz question dealt with the relationship between the Atomic Mass Unit (AMU), Avogadro's number $\left(6.022\times10^{23}\right)$, and the gram.

Avogadro's number is the same as the mol, just like the number $12$ is the same as a "dozen"

i.e. $6.022\times10^{23}=1~mol$ ~~~like~~~ $12=1~dozen$

A dozen is a useful quantity just because it is a good number of things to have (such as doughnuts).

A mol is a useful quantity because it relates the Atomic Mass Unit to the gram. 

$6.022\times10^{23}~AMU=1~mol~AMU=1~g$

This relationship is why atomic weights and molar masses can equally use units of $AMU$ \emph{or} $\dfrac{g}{mol}$

The third question asked:

\vspace{1em}
If a sample weighs $6.34\times10^{26}~AMU$, how many $moles$ of $AMUs$ does it weigh?

\color{blue}
Answer: $\dfrac{6.34\times10^{26}~AMU}{}\left|\dfrac{1~mol}{6.022\times10^{23}}\right.= 1050~mol~AMU$

\color{black}
\vspace{1em}
How many $grams$ does the sample weigh?

\color{blue}
Answer: $\dfrac{1050~mol~AMU}{}\left|\dfrac{1~g}{1~mol~AMU}\right.= 1050~g$

\color{black}
\vspace{1em}\noindent
We can consider the atomic weight of an element to be the conversion factor between grams of a sample and moles of atoms in that sample.

How many $g$ would a $1.75~mol$ sample of \ch{C} weigh?

\color{blue}
Answer: $\dfrac{1.75~mol~\ch{C}}{}\left|\dfrac{12.011~g~\ch{C}}{1~mol~\ch{C}}\right.= 21.0~g~\ch{C}$

\vspace{1em}
\color{black}
How many moles are in a $12.5~g$ sample of \ch{Fe}?

\color{blue}
Answer: $\dfrac{12.5~g~\ch{Fe}}{}\left|\dfrac{1~mol~\ch{Fe}}{55.845~g~\ch{Fe}}\right.= 0.224~mol~\ch{Fe}$
\end{document}