\documentclass[12pt, openany, letterpaper]{memoir}
\usepackage{NotesStyle}
\pagestyle{empty}
%\renewcommand\thesection{\thechapter\Alph{section}}
%\renewcommand\thesubsection{\thesection.\Numeral{subsection}}
\usepackage{xcolor}

\begin{document}
\begin{center}
{\Large Exact Differentials and Natural Variables}
\end{center}

\noindent
Our discussion of natural variables in exact differentials was perhaps a bit pre-mature, relying on the Wikipedia page for thermodynamic functions rather than the textbook.

\noindent
First, let's consider why it is appropriate to consider exact derivatives with respect to different pairs of variables in the first place, using a familiar equation:

$$pV=nRT$$

\noindent
This is an equation with 4 variables: $p$, $V$, $n$, and $T$, but you need only 3 variables to define the system. The fourth variable (whichever you don't have) can be determined from the other three. So, I can take any derivative with respect to pressure, and just replace the pressure with $\dfrac{nRT}{V}$. I can then converted that derivative into one with respect to volume. This is just an application of the chain rule. $U$ is a function of $p$, which is itself a function of $V$, so we can express the derivative in terms of either variable.

\noindent
Entropy and temperature have a similar relationship based on the definition of entropy: 

$$S=\int_{0K}^{T^\prime}\dfrac{\dbar q_{rev}}{T}$$

\noindent
Using this relationship, we can always replace the Entropy with $T$ and vice-versa. The replacement may be messy, but it is possible.

\noindent
So, we can give the exact derivative of $U$ in terms of $V$ and $T$:

$$\mathrm{d}U = \left(\dfrac{\partial U}{\partial V}\right)_{\!\!T}\mathrm{d}V + \left(\dfrac{\partial U}{\partial T}\right)_{\!\!V}\mathrm{d}T$$
 
\noindent
Or, we could give it in terms of $V$ and $S$ instead:

$$\mathrm{d}U = \left(\dfrac{\partial U}{\partial V}\right)_{\!\!S}\mathrm{d}V + \left(\dfrac{\partial U}{\partial S}\right)_{\!\!V}\mathrm{d}S$$

\noindent
Why would we want to do this? Because the partial derivatives in this form are themselves equal to the state variables $-p$ and $T$

$$\left(\dfrac{\partial U}{\partial V}\right)_{\!\!S} = -p \hspace{2em}\mathrm{and}\hspace{2em} \left(\dfrac{\partial U}{\partial S}\right)_{\!\!V}=T$$

\noindent
For the other main thermodynamic potentials ($H$, $G$, and $A$), similar pairs of variables will have this neat property - that the partial derivatives with respect to those pairs of variables are equal to other state variables!

\end{document}