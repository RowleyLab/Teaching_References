\documentclass[12pt,letterpaper]{article}
\usepackage[utf8]{inputenc}
\usepackage[T1]{fontenc}
\usepackage{amsmath}
\usepackage{amsfonts}
\usepackage{amssymb}
\usepackage{hyperref}
\title{Relating the Joule-Thompson Coefficient to the Van der Waals Constants}
\author{Matthew Rowley}
\begin{document}
	\maketitle
	The definition for the Joule-Thompson Coefficient is given in equation \ref{JT_Definition} below
	\begin{equation}\label{JT_Definition}
		\mu\equiv-\frac{\left(\frac{\partial H}{\partial P}\right)_T}{\left(\frac{\partial H}{\partial T}\right)_P}
	\end{equation}

	In order to relate the Joule-Thompson Coefficient to the Van der Waals Constants, we must first take a closer look at enthalpy ($H$). Enthalpy is defined as $H=U+PV$, and is a function of multiple variables. Therefore, we can express its exact differential in multiple ways. We may not have completely covered this yet in lecture, but essentially H is a 2-dimensional function and we can take the partial derivatives with respect to different pairs of variables. For thermodynamic potentials, one pair of variables is called the ``Natural Variables'' because their partial derivatives \emph{are themselves} state variables. For enthalpy, the natural variables are $S$ and $P$, and the exact differential is given below
	\begin{equation}
		\mathrm{d}H=\left(\frac{\partial H}{\partial S}\right)_P\mathrm{d}S + \left(\frac{\partial H}{\partial P}\right)_S\mathrm{d}P
	\end{equation}

	Since $\left(\frac{\partial H}{\partial S}\right)_P=T$ and $\left(\frac{\partial H}{\partial P}\right)_S=V$, the expression simplifies to the expression below. This is the neat result of using the \emph{natural variables}:
	\begin{equation}\label{natural}
		\mathrm{d}H=T\mathrm{d}S + V\mathrm{d}P
	\end{equation}

	However, for $U$ and $H$ it is useful to take derivatives with respect to temperature as well, even though it is not a natural variable for those potentials. For enthalpy the exact derivative is given below:
	\begin{equation}
		\mathrm{d}H=\left(\frac{\partial H}{\partial T}\right)_P\mathrm{d}T + \left(\frac{\partial H}{\partial P}\right)_T\mathrm{d}P
	\end{equation}

	And $\left(\frac{\partial H}{\partial T}\right)_P$ is the constant-pressure heat capacity $c_P$. We can't simplify the other part yet, but so far this gives us:
	\begin{equation}\label{unnatural}
		\mathrm{d}H=c_P\mathrm{d}T + \left(\frac{\partial H}{\partial P}\right)_T\mathrm{d}P
	\end{equation}

	Now we can return to equation \ref{natural}, and differential with respect to P at constant temperature. In practice, this means that we change all our $\mathrm{d}$s into $\partial$s, put $\partial P$ in the denominator on both sides, and now hold T constant. The $\partial P$ in the final term will cancel. The result is:
	
	\begin{equation}
		\left(\frac{\partial H}{\partial P}\right)_T = T \left(\frac{\partial S}{\partial P}\right)_T + V
	\end{equation}

	Now we can plug this into the other exact differential equation for H, equation \ref{unnatural} to get:
	\begin{equation}
		\mathrm{d}H=c_P\mathrm{d}T + \left[T \left(\frac{\partial S}{\partial P}\right)_T + V\right]_T\mathrm{d}P
	\end{equation}

	Here we have a partial derivative of one state variable with respect to another state variable. This is an ideal time to consider using a Maxwell relation, which we also haven't covered. Wikipedia has two pages that are useful references for thermodynamic potentials and Maxwells relations.
	\begin{itemize}
		\item \href{https://en.wikipedia.org/wiki/Thermodynamic_potential}{https://en.wikipedia.org/wiki/Thermodynamic\_potential}
		\item \href{https://en.wikipedia.org/wiki/Maxwell_relations}{https://en.wikipedia.org/wiki/Maxwell\_relations}
	\end{itemize}

	In this instance, we specifically need the Maxwell relation:\\ $\left(\frac{\partial S}{\partial P}\right)_T = - \left(\frac{\partial V}{\partial T}\right)_P$. Using this relation gives us:
	\begin{equation}
		\mathrm{d}H=c_P\mathrm{d}T + \left[V-T \left(\frac{\partial V}{\partial T}\right)_P\right]_T\mathrm{d}P
	\end{equation}

	And remember that $\left[V-T \left(\frac{\partial V}{\partial T}\right)_P\right]_T=\left(\frac{\partial H}{\partial P}\right)_T$, so we can revisit our original definition of the Joule-Thompson coefficient in equation \ref{JT_Definition}:
	\begin{equation}
		\mu\equiv-\frac{\left(\frac{\partial H}{\partial P}\right)_T}{\left(\frac{\partial H}{\partial T}\right)_P} = -\frac{\left[V-T \left(\frac{\partial V}{\partial T}\right)_P\right]_T}{c_P} = \frac{\left[T \left(\frac{\partial V}{\partial T}\right)_P-V\right]_T}{c_P}
	\end{equation}
\end{document}