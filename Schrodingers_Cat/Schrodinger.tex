\documentclass{memoir}
\RequireXeTeX
\usepackage[fleqn]{amsmath}
\usepackage[no-math]{fontspec}
\setmainfont[Ligatures=TeX,Scale=1.25]{Brill}

%Prevent lone lines at page breaks
\clubpenalty = 10000
\widowpenalty = 10000
\thispagestyle{empty}
\usepackage[margin=1.0in]{geometry}

\begin{document}
\title{Where We Go Wrong With Schr\"odinger's Cat}
\author{Matthew Rowley}
\maketitle

As a physical chemist who studies and teaches quantum mechanics, I find it legitimately exciting to see people interested in my field of study. I also find it frustrating to see it interpreted, applied, and extended in careless and inappropriate ways. This frustration is the common curse of experts in all fields, and dwelling on it is an easy gateway to snobbish elitism. However, at risk of becoming an annoying know-it-all, I do want to address some popular misconceptions about the results and consequences of quantum mechanics.

\section*{Popular Modern Views of Quantum Mechanics}
Many popular views of quantum mechanics are based on real experimental results. These experimental results, which were obtained on very small scales (electrons, photons, etc.) are then often extended to apply to the macroscopic world around us. Here are some examples I've heard:

\begin{itemize}
\item Quantum mechanical systems are inherently non-deterministic. That means that we cannot predict the outcome, even with perfect knowledge of the state of the system. Instead, the best we can do is predict the probabilities of observing certain outcomes. The probabilistic nature of quantum mechanical systems on small scales is often cited as evidence for free will in an otherwise deterministic seeming world.
\item When a quantum mechanical system involves several particles, their observed states are linked in a phenomenon called ``quantum entanglement.'' This connection between the particles has been observed to span vast distances, and perturbations to one particle seem to manifest changes in the other particle immediately, regardless of the distance. These sorts of quantum entanglements have been suggested as a means for faster-than-light communication. Quantum entanglement has also been referenced with respect to social connections such as romantic infatuation, or parental intuition.
\item Observers and wave-function collapse.
\end{itemize}

\section*{Foundations of Quantum Mechanics}
postulates

\section*{The Importance of Phase Coherence}
homogeneous broadening

\section*{The Limits of Our Understanding}
Bell's inequality

\end{document}